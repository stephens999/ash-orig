\subsubsection*{Calibration of posterior intervals}

To quantify the effects of errors in estimates of $g$ we examine the calibration of the resulting posterior distributions (averaged over 100 simulations in each Scenario). 
Specifically we examine the empirical coverage of nominal lower 95\% credible bounds for a) all observations; b) significant negative discoveries; c) significant positive discoveries.  We examine only lower bounds because the results for upper bounds follow by 
symmetry (except for the one asymmetric scenario). We separately examine positive and negative discoveries because the lower bound plays a different role
in each case: for negative discoveries the lower bound is typically large and negative and limits how big (in absolute value) 
the effect could be; for positive discoveries the lower bound 
is positive, and limits how small (in absolute value) the effect could be. Intuitively, the lower bound for negative discoveries depends on the accuracy of $g$ in its tail,
whereas for positive discoveries it is more dependent on the accuracy of $g$ in the center.

The results are shown in Table \ref{tab:coverage}.  Most of the empirical coverage rates are in the range 0.92-0.96 for nominal coverage of 0.95, which we view
as adequate for practical applications. The strongest deviations from nominal rates are noted and discussed in the table captions.